\chapter{Leslie's Auto Sales}

Ternary Relationship:
\begin{itemize}
	\item{\textbf{salesperson} : one Salesperson can have zero, one, or many Sales;}
	\item{\textbf{customer} : one Customer can have zero, one, or many Sales;}
	\item{\textbf{automobile} : one Automobile can have zero, one or many Sales;}
	\item{\textbf{sale} : one Sale have one Salesperson, Customer, and Automobile.}
\end{itemize}

\begin{table}[H]
	\begin{center}
		\begin{tabular}{|c|c|}
			\hline
			\textbf{Attribute Name} & \textbf{Attribute Description} \\ \hline
			salespersonID & Salesperson Number (Unique) \\ \hline
			salespersonName & Salesperson Name \\ \hline
			salespersonPhone & Salesperson Telephone \\ \hline
			salespersonYWC & Salesperson Years With Company \\ \hline
		\end{tabular}
		\caption{Salesperson Relation}
	\end{center}
\end{table}

\begin{table}[H]
	\begin{center}
		\begin{tabular}{|c|c|}
			\hline
			\textbf{Attribute Name} & \textbf{Attribute Description} \\ \hline
			customerID & Customer Number (Unique) \\ \hline
			customerName & Customer Name \\ \hline
			customerAddress & Customer Address \\ \hline
		\end{tabular}
		\caption{Customer Relation}
	\end{center}
\end{table}

\begin{table}[H]
	\begin{center}
		\begin{tabular}{|c|c|}
			\hline
			\textbf{Attribute Name} & \textbf{Attribute Description} \\ \hline
			automobileID & Vehicle Identification Number (Unique) \\ \hline
			automobileManufacturer & Vehicle Manufacturer \\ \hline
			automobileModel & Vehicle Model \\ \hline
			automobileYear & Vehicle Year \\ \hline
			automobilePrice & Vehicle Sticker Price \\ \hline
		\end{tabular}
		\caption{Automobile Relation}
	\end{center}
\end{table}

\begin{table}[H]
	\begin{center}
		\begin{tabular}{|c|c|}
			\hline
			\textbf{Attribute Name} & \textbf{Attribute Description} \\ \hline
			saleID & Sale Number (Unique) \\ \hline
			FK\_customerID & Customer Number (Unique) \\ \hline
			FK\_salespersonID & Salesperson Number (Unique) \\ \hline
			FK\_automobileID & Vehicle Identification Number (Unique) \\ \hline
			saleDate & Date of the Sale \\ \hline
			saleNegotiatedPrice & Real Price of the Sale \\ \hline
		\end{tabular}
		\caption{Sale Relation}
	\end{center}
\end{table}

The sale relation describe the interaction of automobile, customer, and salesperson. The value of last purchase of the customer at Leslie's can be discovered by tracking the sale's row which have the most recent date of sale from that customer.